\chapter{Implementierung}
\label{ch:implementierung4}
- Dieses Kapitel beschreibt die Implementierung des (name?) Prototypen den wir zuvor konzipiert haben

\section{Datenschema}
- Ausgehend von den Zuvor in 3.2.1 und 3.2.2 beschriebenen Konzepten soll nun die Implementierung des Datenschemas beschrieben werden.

- H2-DB

- Sql-Script / Datenbankinhalt

- CSV Files bzw. deren Ersatz (in der Spätren Implementierung)

- Mit der erfolgreichen Implementierung der Datenschicht, haben wir nun eine Grundlage mit der wir später unser ML Verfahren sowie unsere Decison Engine füttern können.

\section{Decision-Engine}
- Implementierung Decision Engine 

- Nach Implementierung der Decision Engine muss nun das ML Verfahren implementiert werden, das unsere Entscheidungen verbessern soll

\section{Neuronales-Netzwerk}
- Implementierung nach unseren Anforderungen. 1. Lernen können und 2. Evaluieren können.  

- Erklären inwiefern DeepLearning4J leicht auf andere UseCases angewendet werden kann (Um das Kriterium der Allgemeingültigkeit zu erfüllen)

\section{Optimierung}
- Die Strukturen des NN wurden gesetzt. Nun müssen noch die Lernparameter bestimmt werden.

- Early Stopping using Best Model 

- Trial and Error

- Tabelle aus meiner PPT

- Evaluierungsdaten können auch gelernt werden 
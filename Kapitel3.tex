\chapter{Konzeption}
\label{ch:Konzeption3}

Theoretische Grundlagen geklärt. Zur Erreichung des ersten Teilziels, "'Konzepte ermitteln"', fehlt nun die Konzeption eines Prototypen und die Definition von Anwendungsfällen mit denen ich meine Forschungsfrage prüfen möchte.

\section{Definition von Anwendungsfällen}
\label{sec:Anwendungsfalle3}

Mit dem Prototypen möchte ich anhand von 2 Anwendungsfällen aus dem Finanzbereich prüfen, ob ich die dort modellierten Entscheidungen verbessern kann. Hierzu soll eine vereinfachte Kreditantragsbewertung in DMN modelliert werden. Anschließend möchte ich prüfen ob auch bei einem anderem Anwendungsfall, das selbe Resultat erwartet werden kann. Hierzu soll die Betrugserkennung von Banktransaktionen modelliert werden.   

\subsection{Erkennung von Kreditausfällen}
\label{subsec:Kreditausfallen3}

- Kurze Einordnung des Geschäftszweigs "'Lending"' plus Bedeutung des Themas für den Finanzsektor

- Vorstellen des Datenschemas evtl. tabellarisch (Input und Output Data) 

- Vorstellen des Decision Tables

\subsection{Erkennung von Betrugsversuchen bei Banktransaktionen}
\label{subsec:Banktransaktionen3}

- Kurze Erläuterung von "'Fraud Detection"', sowie dessen Wichtigkeit

- Vorstellen des Datenschemas evtl. tabellarisch (Input und Output Data) 

- Vorstellen des Decision Tables

\section{Identifikation relevanter Themenfelder}
\label{subsec:Themenfelder3}

- Klären der Methodik des Prototyping um These zu beweisen

- Erläuterung des Prototypen und dessen Komponenten 
 
GRAFIK PROTOTYPE AT A GLANCE

\subsection{Lerndaten}
\label{subsec:Lerndaten3}

- Welche Daten werden benötigt um unser Neuronales Netz zu füttern.

- Erklärung des Erzeugungs-Algorithmus 

- Erklären von Feature Scaling

- Erklären von Labels

- Grafik mit Beispiel Datensätzen

\subsection{Evaluationsdaten}
\label{subsec:Evaluationsdaten3}

- Welche Daten werden benötigt um unser Neuronales Netz zu evaluieren.

- Erklärung des Erzeugungs-Algorithmus 

- Erklären von Feature Back Scaling

- Grafik mit Beispiel Datensätzen

\subsection{Decision-Engine}
\label{subsec:Engine3}

- Erklärung der theoretischen Funktionsweise der Decision Engine

- Erklärung des Stellenwerts innerhalb des Prototypen

\subsection{Neuronales-Netzwerk}
- Erklärung des Stellenwerts innerhalb des Prototypen 

- Funktionale Anforderungen: Lernen können sowie Datensatz evaluieren, sprich Output berechnen können

- Sollte leicht Anpassbar auf verschiedene Daten-Modelle bzw. UseCases sein 

